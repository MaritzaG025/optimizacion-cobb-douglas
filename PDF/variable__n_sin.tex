\NeedsTeXFormat{LaTeX2e}
\ProvidesClass{matematicasud}[2020/07/31 Paquete matematicas udistrital]
\LoadClass[12pt, titlepage]{article}
\RequirePackage{xcolor, amssymb, amsmath, amsthm, fancyhdr, hyperref, graphicx, tikz, pgfplots, pgf, tkz-euclide, enumerate, float}

%%CONFIGURACIÓN DE PGF Y TIKZ
\pgfplotsset{width=7cm,compat=1.8}
%\usetkzobj{all}
%\usetikzlibrary{calc, patterns, angles, quotes}
\usepgfplotslibrary{polar}

%%LAS IMAGENES DEBEN ESTAR VECTORIZADAS EN FORMATO PDF
\DeclareGraphicsExtensions{.pdf}

%% ESTILOS DE TEOREMAS
\theoremstyle{plain}
\newtheorem{teorema}{Teorema}
\newtheorem*{teorema*}{Teorema}
\newtheorem{lema}{Lema}
\newtheorem{proposicion}{Proposición}
\newtheorem{corolario}{Corolario}
\theoremstyle{definition}
\newtheorem{definicion}{Definición}
\theoremstyle{remark}
\newtheorem{nota}{Nota}
\newtheorem{ejercicio}{Ejercicio}

%% IDIOMAS
\usepackage[spanish]{babel}

%% MÁRGENES
\usepackage{geometry}
\geometry{letterpaper,tmargin=4cm,bmargin=4cm,lmargin=2cm,
	rmargin=2cm, headheight=1cm,headsep=1cm,footskip=2cm}

%% FUENTES
\usepackage{kpfonts,baskervald}




\usepackage{etoolbox}
\usepackage{hyperref}
\patchcmd{\abstract}{\null\vfil}{}{}{}

\AtBeginDocument{
\setlength{\parskip}{\medskipamount}
\setlength{\parindent}{0pt}
\pagestyle{fancy}
\renewcommand{\headrulewidth}{2pt}
\renewcommand{\footrulewidth}{1pt}
\fancyhf{}
\fancyhead{}
\chead{\leftmark}
\cfoot{Página \thepage}
}


\begin{document}
Para una firma que usa n insumos $x_{1}, ..., x_{n}$ a precios $w_{1}, ..., w_{n}$ por unidad, respectivamente, el costo promedio es:

\[ C(x) = \frac{\sum_{i=1}^{n} w_{i}x_{i} }{f(x_{1}, ..., x_{n})}\]

donde $f = f(x_{1}, ..., x_{n})$  es la función de producción. Las condiciones que se deben satisfacer para minimizar el costo promedio son:

\[ \frac{\partial C}{\partial x_{i}} = \frac{w_{i}f(x_{1}, ..., x_{n}) - f_{x_{i}}(x_{1}, ..., x_{n}) \sum_{i=1}^{n} w_{i}x_{i}}{(f(x_{1}, ..., x_{n}))^{2}} = 0 \]
\\
para todo $i= 1, ..., n$.  Esta ecuación equivale a:

\[ w_{i}f(x_{1}, ..., x_{n}) - f_{x_{i}}(x_{1}, ..., x_{n}) \sum_{i=1}^{n} w_{i}x_{i} = 0 \]

luego de transponer los términos negativos obtenemos que: 

\[ w_{i}f(x_{1}, ..., x_{n}) = f_{x_{i}}(x_{1}, ..., x_{n}) \sum_{i=1}^{n} w_{i}x_{i} \]

para $i = 1, ..., n$. Ahora bien , fijemos $i$ y realicemos el cociente entre las expresiones. Entonces, 

\[ \frac{w_{i}f(x_{1}, ..., x_{n})}{w_{1}f(x_{1}, ..., x_{n})} = \frac{f_{x_{i}}(x_{1}, ..., x_{n}) \sum_{j=1}^{n} w_{j}x_{j}}{f_{x_{1}}(x_{1}, ..., x_{n}) \sum_{j=1}^{n} w_{j}x_{j}}\]
\[\cdots\]
\[ \frac{w_{i}f(x_{1}, ..., x_{n})}{w_{i-1}f(x_{1}, ..., x_{n})} = \frac{f_{x_{i}}(x_{1}, ..., x_{n}) \sum_{j=1}^{n} w_{j}x_{j}}{f_{x_{i-1}}(x_{1}, ..., x_{n}) \sum_{j=1}^{n} w_{j}x_{j}}\]
\[ \frac{w_{i}f(x_{1}, ..., x_{n})}{w_{i+1}f(x_{1}, ..., x_{n})} = \frac{f_{x_{i}}(x_{1}, ..., x_{n}) \sum_{j=1}^{n} w_{j}x_{j}}{f_{x_{i+1}}(x_{1}, ..., x_{n}) \sum_{j=1}^{n} w_{j}x_{j}}\]
\[\cdots\]
\[ \frac{w_{i}f(x_{1}, ..., x_{n})}{w_{n}f(x_{1}, ..., x_{n})} = \frac{f_{x_{i}}(x_{1}, ..., x_{n}) \sum_{j=1}^{n} w_{j}x_{j}}{f_{x_{n}}(x_{1}, ..., x_{n}) \sum_{j=1}^{n} w_{j}x_{j}}\]

simplificando obtenemos que,

\[\frac{w_{i}}{w_{1}} = \frac{f_{x_{i}}(x_{1}, ..., x_{n})}{f_{x_{1}}(x_{1}, ..., x_{n})}, \ \cdots, \ \frac{w_{i}}{w_{i-1}} = \frac{f_{x_{i}}(x_{1}, ..., x_{n})}{f_{x_{i-1}}(x_{1}, ..., x_{n})}, \ \frac{w_{i}}{w_{i+1}} = \frac{f_{x_{i}}(x_{1}, ..., x_{n})}{f_{x_{i+1}}(x_{1}, ..., x_{n})}, \ \cdots, \ \frac{w_{i}}{w_{n}} = \frac{f_{x_{i}}(x_{1}, ..., x_{n})}{f_{x_{n}}(x_{1}, ..., x_{n})}\]

para cada $i = 1, ..., n$. Entonces, las cantidades de insumos que el productor debe usar para minimizar su costo promedio son aquellas para las cuales la relación entre las productividades marginales es igual a la relación entre los precios de los insumos de producción. Así, si el productor vende su producto a P por unidad, los beneficios están dados por:

\[ \Pi(x_{1}, ..., x_{n}) = Pf(x_{1}, ..., x_{n}) - \sum_{i=1}^{n} w_{i}x_{i} \]
\end{document}